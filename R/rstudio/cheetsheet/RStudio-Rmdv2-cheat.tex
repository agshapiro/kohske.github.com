\documentclass[10pt,landscape]{ltjarticle}
\usepackage{multicol}
\usepackage{calc}
\usepackage{ifthen}
\usepackage[landscape]{geometry}
\usepackage{hyperref}
\usepackage{applekeys}
\usepackage{luatexja-fontspec}
\usepackage{framed}
\setmainfont[Ligatures=TeX]{TeXGyreHeros}
\geometry{top=1cm,left=1cm,right=1cm,bottom=1cm} 
\pagestyle{empty}
\makeatletter
\renewcommand{\subsection}{\@startsection{subsection}{2}{0mm}%
                                {-1explus -.5ex minus -.2ex}%
                                {0.5ex plus .2ex}%
                                {\normalfont\normalsize\bfseries}}
\makeatother
\setcounter{secnumdepth}{0}
\setlength{\parindent}{0pt}
\setlength{\parskip}{0pt plus 0.5ex}

\begin{document}

\raggedright
\footnotesize
\begin{multicols}{3}

\setlength{\premulticols}{1pt}
\setlength{\postmulticols}{1pt}
\setlength{\multicolsep}{1pt}
\setlength{\columnsep}{2pt}

\begin{center}
     \Large{\textbf{RStudioキーボードショートカット\\(完全版)}}
\end{center}

\today

\subsection{コンソール}
\begin{tabular}{p{30mm}p{25mm}p{25mm}}
カーソルをコンソールに & \ctlkey 2 & \ctlkey 2 \\
コンソールを消去 & \ctlkey L & \cmdkey\ L \\
カーソルを行頭に & Home & \cmdkey\ Left \\
カーソルを行末に & End & \cmdkey\ Right \\
履歴 & Up/Down & Up/Down \\
履歴をポップアップ & \ctlkey Up & \cmdkey\ Up \\
実行中のコマンドを停止 & Esc & Esc \\
作業フォルダの変更 & \ctlkey \shiftkey\ H & \ctlkey \shiftkey\ H
\end{tabular}

\subsection{編集}
\begin{tabular}{p{30mm}p{25mm}p{25mm}}
取り消す & \ctlkey Z & \cmdkey\ Z \\
やり直す & \ctlkey \shiftkey\ Z & \cmdkey\ \shiftkey\ Z \\
カット & \ctlkey X & \cmdkey\ X \\
コピー & \ctlkey C & \cmdkey\ C \\
ペースト & \ctlkey V & \cmdkey\ V \\
全て選択 & \ctlkey A & \cmdkey\ A \\
単語に移動 & \ctlkey Left/Right & \optkey\ Left/Right \\
先頭/末尾に & \ctlkey Home/End \par \ctlkey Up/Down & \cmdkey\ Home/End or \cmdkey\ Up/Down \\
行を削除 & \ctlkey D & \cmdkey\ D \\
選択 & \shiftkey\ [Arrow] & \shiftkey\ [Arrow] \\
単語を選択 & \ctlkey \shiftkey\ Left/Right & \optkey\ \shiftkey\ Left/Right \\
行頭まで選択 & \optkey\ \shiftkey\ Left & \cmdkey\ \shiftkey\ Left \\
行末まで選択 & \optkey\ \shiftkey\ Right & \cmdkey\ \shiftkey\ Right \\
1ページ分を選択 & \shiftkey\ PageUp/Down & \shiftkey\ PageUp/Down \\
先頭/末尾まで選択 & \ctlkey \shiftkey\ Home/End \par \shiftkey\ \optkey\ Up/Down & \cmdkey\ \shiftkey\ Up/Down \\
左の単語を削除 & \ctlkey Backspace & \optkey\ Backspace \\
右の単語を削除 &  & \optkey\ Delete \\
行末まで削除 &  & \ctlkey K \\
行頭まで削除 &  & \optkey\ Backspace \\
インデント & Tab (行頭で) & Tab (行頭で) \\
インデント解除 & \shiftkey\ Tab & \shiftkey\ Tab \\
行頭までヤンク & \ctlkey U & \ctlkey U \\
行末までヤンク & \ctlkey K & \ctlkey K \\
ヤンクを挿入 & \ctlkey Y & \ctlkey Y \\
\verb|<-|を挿入 & \optkey\ - & \optkey\ - \\
カーソル位置の関数のヘルプを表示 & F1 & F1 \\
カーソル位置の関数のコードを表示 & F2 & F2
\end{tabular}

\subsection{補完}
\begin{tabular}{p{30mm}p{30mm}p{20mm}}
補完 & Tab / \ctlkey Space & Tab / \cmdkey\ Space \\
候補を選択 & Up/Down & Up/Down \\
補完を実行 & \returnkey\ /Tab/Right & \returnkey\ /Tab/Right \\
補完を中止 & Esc & Esc
\end{tabular}


  \subsection{エディタ}
\begin{tabular}{p{30mm}p{30mm}p{20mm}}
ファイル/関数を検索 & \ctlkey . & \ctlkey . \\
カーソルをエディタに & \ctlkey 1 & \ctlkey 1 \\
新規ファイル & \ctlkey \shiftkey\ N & \cmdkey\ \shiftkey\ N \\
新規ファイル(Chrome) & \ctlkey \optkey\ \shiftkey\ N & \cmdkey\ \shiftkey\ \optkey\ N \\
ファイルを開く & \ctlkey O & \cmdkey\ O \\
ファイルを保存 & \ctlkey S & \cmdkey\ S \\
ファイルを閉じる & \ctlkey W & \cmdkey\ W \\
ファイルを閉じる \par (Chrome) & \ctlkey \optkey\ W & \cmdkey\ \optkey\ W \\
全てのファイルを閉じる & \ctlkey \shiftkey\ W & \cmdkey\ \shiftkey\ W \\
HTMLをプレビュー & \ctlkey \shiftkey\ Y & \cmdkey\ \shiftkey\ Y \\
ファイルをKnit() & \ctlkey \shiftkey\ K & \cmdkey\ \shiftkey\ K \\
ノートブックを作成 & \ctlkey \shiftkey\ K & \cmdkey\ \shiftkey\ K \\
PDFを作成 & \ctlkey \shiftkey\ I & \cmdkey\ \shiftkey\ I \\
チャンクを挿入 & \ctlkey \optkey\ I & \cmdkey\ \optkey\ I \\
コードブロックを挿入 & \ctlkey \shiftkey\ R & \cmdkey\ \shiftkey\ R \\
現在行/選択範囲を評価 & \ctlkey \returnkey & \cmdkey\ \returnkey \\
前回の範囲を再評価 & \ctlkey \shiftkey\ P & \cmdkey\ \shiftkey\ P \\
ファイルを評価 & \ctlkey \optkey\ R & \cmdkey\ \optkey\ R \\
先頭から現在行まで評価 & \ctlkey \optkey\ B & \cmdkey\ \optkey\ B \\
現在行から最後まで評価 & \ctlkey \optkey\ E & \cmdkey\ \optkey\ E \\
関数定義を評価 & \ctlkey \optkey\ F & \cmdkey\ \optkey\ F \\
コードブロックを評価 & \ctlkey \optkey\ T & \cmdkey\ \optkey\ T \\
前のチャンクを評価 & \ctlkey \optkey\ P & \cmdkey\ \optkey\ P \\
現在のチャンクを評価 & \ctlkey \optkey\ C & \cmdkey\ \optkey\ C \\
次のチャンクを評価 & \ctlkey \optkey\ N & \cmdkey\ \optkey\ N \\
ファイルを\verb|source| & \ctlkey \shiftkey\ O & \cmdkey\ \shiftkey\ O \\
現在のファイルを\verb|source| & \ctlkey \shiftkey\ S & \cmdkey\ \shiftkey\ S \\
エコー付きで\verb|source| & \ctlkey \shiftkey\ \returnkey & \cmdkey\ \shiftkey\ \returnkey \\
選択範囲を折り畳み & \optkey\ L & \cmdkey\ \optkey\ L \\
選択範囲を折り畳み解除 & \shiftkey\ \optkey\ L & \cmdkey\ \shiftkey\ \optkey\ L \\
全て折り畳み & \optkey\ O & \cmdkey\ \optkey\ O \\
全て折り畳み解除 & \shiftkey\ \optkey\ O & \cmdkey\ \shiftkey\ \optkey\ O \\
行番号に移動 & \shiftkey\ \optkey\ G & \cmdkey\ \shiftkey\ \optkey\ G \\
関数/チャンクにジャンプ & \shiftkey\ \optkey\ J & \cmdkey\ \shiftkey\ \optkey\ J \\
タブの切り替え & \ctlkey \optkey\ Down & \ctlkey \optkey\ Down \\
左/右のタブ & Win: \ctlkey \optkey\ Left/Right \par Linux: \ctlkey PageUp/Down & \ctlkey \optkey\ Left/Right \\
左端/右端のタブ & \ctlkey \shiftkey\ \optkey\ Left/Right & \ctlkey \shiftkey\ \optkey\ Left/Right \\
次へ移動 & \ctlkey F9 & \cmdkey\ F9 \\
前へ移動 & \ctlkey F10 & \cmdkey\ F10 \\
選択範囲から関数を作成 & \ctlkey \optkey\ X & \cmdkey\ \optkey\ X \\
選択範囲から変数を作成 & \ctlkey \optkey\ V & \cmdkey\ \optkey\ V \\
行をインデント & \ctlkey I & \cmdkey\ I \\
行/範囲をコメント & \ctlkey \shiftkey\ C & \cmdkey\ \shiftkey\ C \\
コメントを整形 & \ctlkey \shiftkey\ / & \cmdkey\ \shiftkey\ / \\
カーソル前後を入れ替え &  & \ctlkey T \\
行/範囲を上下に移動 & \optkey\ Up/Down & \optkey\ Up/Down \\
行/範囲を上下にコピー & \shiftkey\ \optkey\ Up/Down & \cmdkey\ \optkey\ Up/Down \\
対応する括弧に移動 & \ctlkey P & \ctlkey P \\
検索/置換 & \ctlkey F & \cmdkey\ F \\
次を検索 & Win:F3, Linux:\ctlkey G & \cmdkey\ G \\
前を検索 & Win:\shiftkey\ F3, Linux:\ctlkey \shiftkey\ G & \cmdkey\ \shiftkey\ G \\
選択範囲で検索 & \ctlkey F3 & \cmdkey\ E \\
置換して次に & \ctlkey \shiftkey\ J & \cmdkey\ \shiftkey\ J \\
複数のファイルを検索 & \ctlkey \shiftkey\ F & \cmdkey\ \shiftkey\ F \\
スペルチェック & F7 & F7 
\end{tabular}

\subsection{表示}
\begin{tabular}{p{30mm}p{30mm}p{20mm}}
エディタにフォーカス & \ctlkey 1 & \ctlkey 1 \\
コンソールにフォーカス & \ctlkey 2 & \ctlkey 2 \\
ヘルプにフォーカス & \ctlkey 3 & \ctlkey 3 \\
履歴タブを表示 & \ctlkey 4 & \ctlkey 4 \\
ファイルタブを表示 & \ctlkey 5 & \ctlkey 5 \\
プロットタブを表示 & \ctlkey 6 & \ctlkey 6 \\
パッケージタブを表示 & \ctlkey 7 & \ctlkey 7 \\
環境タブを表示 & \ctlkey 8 & \ctlkey 8 \\
Git/SVNタブを表示 & \ctlkey 9 & \ctlkey 9 \\
ビルドタブを表示 & \ctlkey 0 & \ctlkey 0 \\
PDFプレビューと同期 & \ctlkey F8 & \cmdkey\ F8 \\
ショートカットを表示 & \optkey\ \shiftkey\ K & \optkey\ \shiftkey\ K
\end{tabular}

\subsection{ビルド・パッケージ・VCS}
\begin{tabular}{p{30mm}p{30mm}p{20mm}}
ビルドしてリロード & \ctlkey \shiftkey\ B & \cmdkey\ \shiftkey\ B \\
全てをロード & \ctlkey \shiftkey\ L & \cmdkey\ \shiftkey\ L \\
パッケージテスト & \ctlkey \shiftkey\ T & \cmdkey\ \shiftkey\ T \\
パッケージテスト(web) & \ctlkey \optkey\ F7 & \cmdkey\ \optkey\ F7 \\
パッケージチェック & \ctlkey \shiftkey\ E & \cmdkey\ \shiftkey\ E \\
ドキュメントの作成 & \ctlkey \shiftkey\ D & \cmdkey\ \shiftkey\ D \\
ドキュメントの差分 & \ctlkey \optkey\ D & \ctlkey \optkey\ D \\
コミット & \ctlkey \optkey\ M & \ctlkey \optkey\ M \\
差分ビューのスクロール & \ctlkey Up/Down & \ctlkey Up/Down \\
Stage/Unstage (Git) & Spacebar & Spacebar \\
Stage/Unstageして次に移動 (Git) & \returnkey\  & \returnkey\ 
\end{tabular}

\subsection{プロット}
\begin{tabular}{p{30mm}p{30mm}p{20mm}}
前のプロット & \ctlkey \shiftkey\ PageUp & \cmdkey\ \shiftkey\ PageUp \\
次のプロット & \ctlkey \shiftkey\ PageDown & \cmdkey\ \shiftkey\ PageDown
\end{tabular}

\subsection{セッション・デバッグ}
\begin{tabular}{p{30mm}p{30mm}p{20mm}}
セッションの終了 & \ctlkey Q & \cmdkey\ Q \\
Rを再起動 & \ctlkey \shiftkey\ F10 & \cmdkey\ \shiftkey\ F10 \\
ブレークポイントの設定 & \shiftkey\ F9 & \shiftkey\ F9 \\
次の行を評価 & F10 & F10 \\
評価を続ける & \shiftkey\ F5 & \shiftkey\ F5 \\
デバッグの中止 & \shiftkey\ F8 & \shiftkey\ F8
\end{tabular}

\begin{framed}
左はWindows/Linux、右はMac OS X

\begin{description}
\setlength{\parskip}{0cm} 
\setlength{\itemsep}{1pt}
\item[\ctlkey] Control
\item[\cmdkey] コマンド
\item[\shiftkey] シフト
\item[\optkey] Alt (Win/Linux) / Option (Mac)
\end{description}
\end{framed}

\rule{0.3\linewidth}{0.25pt}

\scriptsize
Copyright \copyright\ 2014 @kohske

\end{multicols}
\end{document}